\secrel{x86os: простая операционная система\\ для компьютера на i386
процессоре}\secdown

\secrel{Ресурсы}

\url{http://wiki.osdev.org}

\bigskip

\href{http://www.frolov-lib.ru/bsp.html}{Библиотека системного программиста
\copyright\ Александр Фролов, Григорий Фролов}:

\begin{itemize}[nosep]
  \item \href{http://www.frolov-lib.ru/books/bsp/v01a/index.html}{Операционная
  система MS-DOS. Том 1, книги 1-2}
  \item \href{http://www.frolov-lib.ru/books/bsp/v01b/index.html}{Операционная
  система MS-DOS. Том 1, книга 3}
  \item \href{http://www.frolov-lib.ru/books/bsp/v02/index.html}{Аппаратное
  обеспечение IBM PC. Том 2, книга 1}
  \item \href{http://www.frolov-lib.ru/books/bsp/v03/index.html}{Программирование видеоадаптеров CGA, EGA и VGA}
  \item \href{http://www.frolov-lib.ru/books/bsp/v06/index.html}{Защищенный
  режим процессоров Intel 80286/80386/80486. Том 4}
\end{itemize}

\bigskip

Графика: \bigskip

\begin{itemize}[nosep]

\item 
Уилтон Р. 

Видеосистемы персональных компьютеров IBM РС и РS/2.

Радио и связь, 1994

\item

\href{http://ict.informika.ru/ft/004761/vasilev.pdf}{С.А.Васильев
Программирвание видеосистем, ТГТУ, 2003}

\item

Е.В. Шикин, А.В. Боресков

\begin{itemize}[nosep]
\item Компьютерная графика. Динамика, реалистические изображения.
Диалог-МИФИ, 1996
\item Компьютерная графика. Полигональные модели. Диалог-МИФИ, 2001
\item Начала компьютерной графики. Диалог-МИФИ, 1993
\end{itemize}

\end{itemize}

\secrel{Структура}

\begin{itemize}
  \item кросс-компилятор
  \item загрузчик
  \item микроядро
  \item драйвера
  \item библиотеки
  \item прикладные программы
\end{itemize}

\secrel{Процесс запуска}

\begin{enumerate}
  \item BIOS
  \item boot0 первый сектор загрузочного диска, считывает boot1
  \item boot1 основная часть загрузчика, в файле на загрузочном диске
  \item ядро ОС
  \item запуск драйверов
  \item запуск системных демонов (сервисы)
  \item запуск GUI (если есть) или пользовательской консоли
\end{enumerate}

\secrel{Сборка кросс-компилятора}

\begin{verbatim}
git clone --depth=1 -o gh https://github.com/ponyatov/x86os.git
cd x86os 
make dirs
make gz

make cross
\end{verbatim}

\lstx{cfg.mk}{}{x86os/cfg.mk}{mk}
\lstx{mk/versions.mk}{}{x86os/mk/versions.mk}{mk}
\lstx{mk/packages.mk}{}{x86os/mk/packages.mk}{mk}
\lstx{mk/dirs.mk}{}{x86os/mk/dirs.mk}{mk}
\lstx{mk/src.mk}{}{x86os/mk/src.mk}{mk}
\lstx{mk/cross.mk}{}{x86os/mk/cross.mk}{mk}
\lstx{mk/cclibs.mk}{}{x86os/mk/cclibs.mk}{mk}

Как вы можете заметить, структура файлов и каталогов похожа на
\prog{azlin} /\ref{azlin}/.

\secrel{Формат ELF32}

Собранный кросс-компилятор генерирует исполняемые файлы, т.е. файл ядра, в
формате \termdef{ELF}{ELF}. Содержание можно посмотреть в файле
\file{kernel.objdump}, который создается при выполнении команды 

\begin{verbatim}
make kernel
\end{verbatim}

\begin{tabular}{l l l}
формат файла & elf32-i386 \\
архитектура & i386 \\
HAS\_SYMS & в файл включена отладочная информация \\& об идентификаторах
(``символах'')\\
start address & 0x00100000 & стартовый адрес загрузки ядра 1\,Мб\\
\end{tabular}

\bigskip
Бинарный код делится на \termdef{секции}{секция ELF}, или
\termdef{сегменты}{сегмент}:

\begin{tabular}{l l}
.text & машииный код программы \\
.rodata & данные: константы \\
.eh\_frame &\\
.data & данные: инициализированные массивы, строки \\
.bss & данные: пустые массивы под которые выделяется память при старте \\
.comment &\\
\end{tabular}

\bigskip
\begin{tabular}{l l}
CONTENTS & \\
ALLOC & \\
LOAD & \\
READONLY & \\
CODE & \\
DATA & \\
\end{tabular}

\lstx{kernel.objdump}{}{x86os/kernel.objdump}{objdump}

\secrel{multiboot}

\href{http://www.syslinux.org/}{загрузчик SysLinux}

\href{http://www.gnu.org/software/grub/manual/multiboot/multiboot.html}{Специфиация
Multiboot 0.6.96}

\secrel{Драйвера}\secdown

\secrel{\prog{vga}: текстовая консоль VGA 80$\times$25}

\lstx{include/driver/vga.h}{}{x86os/include/driver/vga.h}{c}
\lstx{driver/vga.c}{}{x86os/driver/vga.c}{c}

\secrel{\prog{kbd}: клавиатура}

\secrel{\prog{ide}: жесткий диск IDE}\secdown
\secrel{\prog{fatfs}: файловая система FAT16}
\secup

\secup

\secup
