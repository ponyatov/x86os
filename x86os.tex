\secrel{x86os: простая операционная система\\ для компьютера на i386
процессоре}\secdown

\secrel{Ресурсы}

\href{http://wiki.osdev.org}{OS Dev wiki}

\bigskip

\href{http://www.frolov-lib.ru/bsp.html}{Библиотека системного программиста
\copyright\ Александр Фролов, Григорий Фролов}:

\begin{itemize}
  \item \href{http://www.frolov-lib.ru/books/bsp/v01a/index.html}{Операционная
  система MS-DOS. Том 1, книги 1-2}
  \item \href{http://www.frolov-lib.ru/books/bsp/v01b/index.html}{Операционная
  система MS-DOS. Том 1, книга 3}
  \item \href{http://www.frolov-lib.ru/books/bsp/v02/index.html}{Аппаратное
  обеспечение IBM PC. Том 2, книга 1}
  \item \href{http://www.frolov-lib.ru/books/bsp/v03/index.html}{Программирование видеоадаптеров CGA, EGA и VGA}
  \item \href{http://www.frolov-lib.ru/books/bsp/v06/index.html}{Защищенный
  режим процессоров Intel 80286/80386/80486. Том 4}
\end{itemize}

\bigskip

Уилтон Р. 

Видеосистемы персональных компьютеров IBM РС и РS/2.

Радио и связь, 1994

\bigskip

\href{http://ict.informika.ru/ft/004761/vasilev.pdf}{С.А.Васильев
Программирвание видеосистем, ТГТУ, 2003}

\bigskip

Е.В. Шикин, А.В. Боресков

\begin{itemize}
\item Компьютерная графика. Динамика, реалистические изображения.
Диалог-МИФИ, 1996
\item Компьютерная графика. Полигональные модели. Диалог-МИФИ, 2001
\item Начала компьютерной графики. Диалог-МИФИ, 1993
\end{itemize}

\secrel{Структура}

\begin{itemize}
  \item кросс-компилятор
  \item загрузчик
  \item микроядро
  \item драйвера
  \item библиотеки
  \item прикладные программы
\end{itemize}

\secrel{Процесс запуска}

\secrel{Сборка кросс-компилятора}

\lstx{mk/versions.mk}{}{x86os/mk/versions.mk}{mk}
\lstx{mk/dirs.mk}{}{x86os/mk/dirs.mk}{mk}
\lstx{mk/src.mk}{}{x86os/mk/src.mk}{mk}
\lstx{mk/cross.mk}{}{x86os/mk/cross.mk}{mk}

\secrel{multiboot}

\href{http://www.syslinux.org/}{загрузчик SysLinux}

\href{http://www.gnu.org/software/grub/manual/multiboot/multiboot.html}{Специфиация
Multiboot 0.6.96}

\secrel{Драйвера}\secdown

\secrel{\prog{vgacon}: текстовая консоль VGA 80$\times$25}

\secrel{\prog{kbd}: клавиатура}

\secrel{\prog{ide}: жесткий диск IDE}\secdown
\secrel{\prog{fatfs}: файловая система FAT16}
\secup

\secup

\secup
